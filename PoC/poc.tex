% !TEX TS-program = xelatex
% !TEX encoding = UTF-8

\documentclass[11pt]{article}

\usepackage[margin=2.25cm]{geometry}
\usepackage{fontspec} 
\usepackage{textcomp}
\defaultfontfeatures{Mapping=tex-text} 
\usepackage{xunicode} 
\usepackage{xltxtra}
\usepackage{fancyhdr} 
\usepackage{xspace}
\usepackage[usenglish]{babel}

\setmainfont{Charis SIL} 
\setsansfont{DejaVu Sans}
\setmonofont{DejaVu Sans Mono}

\newfontfamily\titlefont[Scale=2]{Charis SIL}
\newfontfamily\subtitlefont[Scale=1.2]{Charis SIL}

% other LaTeX packages.....
\geometry{letterpaper} 
\usepackage[parfill]{parskip} 

\usepackage{graphicx} 
\usepackage{color}

%% Additional commands
\newcommand\codedx{{\color{blue}Code~Dx}\xspace}
\newcommand\MS{Microsoft\texttrademark\xspace}
\newcommand\python{Python\xspace}
\newcommand\java{Java\xspace}
\newcommand\csharp{C\#\xspace}

% Page styling
\pagestyle{fancy}

\fancyhead{}
\fancyfoot{}
\fancyfoot[le,ro]{Page \thepage}
\fancyfoot[re,lo]{\includegraphics[scale=0.2]{CodeDx-logo.png}}
\renewcommand\headrulewidth{0mm}

%%%%%
%%%%% Begin Document
%%%%%
\begin{document}
\begin{center}
  {\titlefont Code Dx Proof of Concept (PoC) Suggestions}\\
  {\subtitlefont Vincent M. Hopson  -- Field Applications Engineer at Code Dx}\\[5mm]
\end{center}

Here are a few items that can generally be used to create a compelling Proof of Concept at your site.  All items
listed here should be considered to be optional.  We want to provide a framework to impart as much relevant 
knowledge about \codedx as possible.

%%%%%
\section{Static Analysis Tools (SAST)}

\codedx recommends that at least one commercial tool be used for your SAST testing.  A big part of using
our solution is the correlation between different tools.  By using a commercial tool, and \codedx's internal
orchestration, you will be able to see findings that have been detected by multiple tools.  This reduces the
possibility of {\color{red}False Positives} considerably.

%%%%%
\section{Dynamic Analysis (DAST)}

Normally, some form of dynamic testing is performed regardless of the application.  \codedx can receive the
test results directly from tools like Portswigger's Burpsuite Professional, or the OWASP Zed Attack Proxy
or ZAP.  The latter is free, and the former is very low cost (at the time of this writing, about \$300 USD).

We support plug-ins for both tools, and can take results from the tools in other formats in case you have
unalterable copies.

It is possible to get results from other tools into \codedx, and we would be happy to help get some of those
into our product.  This exercise helps with later automation needs you may have.

%%%%%
\section{Developer Desktop Integrations}

It is essential that ease of use be addressed in almost any setting.  Especially if working with developers to 
address various findings.  \codedx supports integrations into the most popular developer environments:
\begin{itemize}
	\item \MS Visual Studio
	\item Eclipse
	\item IntelliJ
\end{itemize}

Installation of the \codedx integrations helps demonstrate how easy it is to gather results directly on the
developer's desktop.  By the security Engineer marking findings in \codedx to indicate the person responsible,
an easy search can create a list of issues to be addressed in the desktop environment.

%%%%%
\section{Continuous Integration (CI) Installation}

\codedx supports many of the current Continuous Integration systems that are in common use.  By
taking a moment to install the plug-in, the CI environment can use \codedx to determine the health of
a build.  This step could be used to provide the builder important security information, and automatically
``break the build``; preventing potentially vulnerable code a path to your customers.

%%%%%
\section{Dashboards}

This tends to be more of a long term item as  dashboard updating is performed once a day (usually
early morning server time).  By organizing the projects into sub-projects, it is possible to understand
your security posture at a glance.

Data for the dashboards in \codedx is available on a per-project basis.  A switch in the top right corner
of any dashboard display controls the available data.  When sub-projects are populated with
findings, the switch enables statistics from the current project to be displayed in isolation.  Toggling the
switch takes all of the available data from the current project and all sub-projects and displays it.

A valuable tool for assessing your progress in finding reduction and mitigation.

%%%%%
\section{Report Generation or Custom Reports}

Data accumulation is a tough task, but reporting on what you find effectively can be nearly as big a
challenge.  Information found inside of \codedx can be used in a myriad of ways to help protect your
code, or potentially your reputation outside of your organization.  Reports can be generated inside of
the tool as a number of formats with PDF probably being the most prevalent.

Using the Apache Format Object Processor (FOP) inside of \codedx, we can generate alternate
reports that cover the entire spectrum of intended audiences.  From a risk assessment report to the CEO 
to a code block to be given to an Engineer.  We can help you build effective reports that are black and
white, or colorful descriptions of your security and/or quality stance.

%%%%%
\section{Connect to Your Engineers}

A couple of tight integrations help you to connect appropriate data to your Engineers.  Using Git to read
your repository and collect the most recent code is perhaps the most obvious.  Any Git based system
can be addressed.

Another common integration tool is to track various findings inside of Atlassian's Jira.  \codedx can be
configured to push data to Jira on command, or automatically push tickets into Jira when analyses complete.
This is a great way to show value by integrating Jira into the PoC workflow.

%%%%%
\section{Penetration Test (PEN) Results}

If you have results you can use from your PEN Testing in an electronic format, we can create a script
(\python is my preferred scripting language, but we can use the language of your choice) to put those
results into \codedx.  This operation will show you how to gather the results into \codedx and what should
be captured for correlations.

Note that an interface inside of \codedx will allow entry of PEN test results, and is a good display of
the information that would be needed for good correlation.

%%%%%
\section{Threat Modeling}

It is possible to ingest information into \codedx from any phase of your development pipeline.  \MS
provides a free threat modeler that can be used to create results that \codedx can ingest directly.  This
tool only requires a knowledge of your application (or a hypothetical one) to create useful information.

%%%%%
\section{Hybrid Analysis}

If your team is using \java or \csharp, it is possible to configure \codedx to perform a deeper examination of
your testing results.  This ``hybrid`` analysis can correlate SAST findings with your DAST operations.

There are two brands of \emph{hybrid} analysis; trace based analysis (\java only), and traceless (both
\java and \csharp).  Setting these up is relatively easy, and can provide a much deeper window into how
static findings relate to detected findings.  These tend to have a \emph{very low} {\color{red} False Positive} rate.

\end{document}
